\documentclass[11pt]{article}

\usepackage[margin=1in]{geometry}
\usepackage{graphicx}
\usepackage{gensymb}
\usepackage{hyperref}

\begin{document}

\title{Formal methods at work 2018\\Essay about the class}

%\maketitle

\section*{Renan Greca}

The formal methods class was a very interesting start to my studies at GSSI. While the subject is related to courses I took during undergrad, it showed a new way to describe and understand computational problems and programs.

I liked working with the operational semantic for a somewhat abstract representation of a programming language or compiler. I like the idea of explaining complex structures (like a while loop) using simpler structures already defined. It feels like cracking a puzzle, while also helping us to better understand things that seem so familiar.

I also really enjoyed working with UPPAAL, which also provides a puzzle-like approach to modelling solutions to certain problems. It's something I will consider using in the future and might provide valuable insight. I believe this is also the take-home message of the course: some problems might be easier to understand and solve using different types of semantics or representations. Instead of jumping straight to coding, thinking of the problem somewhat abstractly can be useful.

The negative point, to me, was that little time was dedicated to denotational semantics. I believe I was able to understand operational semantics very well, but some aspects of denotational semantics were considerably harder to grasp and I don't feel as confident using it, but maybe it's because I am more of a programmer than a mathematician. Regardless, it gave me the foundation to learn more about it if I ever feel inclined to do so.

\section*{Konstantin Prokopchik}
\subsection*{Take-away}

I think what I have got from this course is a base notion of different semantics that can be used to describe almost everything in the computer science world, which is amazing. And it's important to know that different semantics can be equal with certain conditions, and that could be helpful.

\subsection*{What I liked}

It is a pretty easy question, because I liked all of it. You have started my education here, in GSSI, on a very high level, compared to the quality of representing material in my own University in Novosibirsk. The English pronunciation was pure and easy to understand, and I can now see that not a lot of Italian teachers here are able to do that. The material was not that hard, but nevertheless, presented very well. I also liked, and this is important to me, that the lectures were delivered through the blackboard, and not by slides. I am used to writing the lectures down at my University, we barely had any slides until my master's program. I think it helps the material to settle much better in our heads.

\subsection*{What I disliked}

Can not really say much here, but for this passage not to be empty, I might mention, that sometimes you spent more time than I thought needed staying on a certain subject, I can't point to the exact topics, but it happened sometimes. I am not sure if this is even a problem, just a certain notion.

\subsection*{Conclusion}

To conclude I would Like to say that I'm very happy to be here (and happy about this course so far) even though I don't look it.

\end{document}
